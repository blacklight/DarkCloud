\chapter*{Abstract}
\addcontentsline{toc}{chapter}{Abstract}
\markboth{Abstract}{}

Nel mondo attuale la sicurezza delle organizzazioni pi� importanti risiede nel
livello di protezione delle proprie informazioni.

Ci sono molti metodi per proteggere documenti informatici. Alcuni prevedono che il dato sia
protetto fisicamente, altri che sia protetto da password o ancora sia raggiungibile dalla rete
ma dietro solidi FIREWALL.

Nella societ� attuale per� non ci si pu� permettere di avere importanti moli di dati
che non siano fruibili da persone autorizzate sparse per tutto il globo.
Quindi la disponibilit� in rete � un requisito fondamentale, che espone per� a gravi 
rischi i nostri dati.
Riporre le proprie speranze in soluzioni monolitiche come pesanti crittografie o 
enti che raccolgono molti dati si � rivelato nel tempo una soluzione non vincente.

Dovendo destreggiarci nella rete conviene sfruttare quelli che sono i suoi punti forti, 
come ad esempio l'enorme quantit� di nodi che la compongono e la semplicit� di scambio dei dati.
E' proprio per questo motivo che abbiamo pensato di sviluppare un software che frammenti il nostro
dato e lo vada a salvare in vari nodi sparsi sul web, ma non nodi appartenenti a reti note di cui si conoscono i nodi 
come quelle dedicate utilizzate dai servizi di {\itshape file sharing}.
Perch� questo esporrebbe troppo il sistema, invece siamo ricorsi ad un tipo di rete molto utilizzato nel'underground digitale 
una rete che viene definita {\itshape darknet}.
In questo tipo di rete solo chi ne fa parte sa da chi � composta la rete. 






\section*{Obiettivi del software}
