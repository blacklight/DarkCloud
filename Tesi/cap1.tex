\chapter{Introduzione}
A livello globale si pu� osservare una crescita esponenziale in termini di numero e gravit� di attacchi informatici. Il problema per� non tocca solo le grandi aziende, questo era vero una volta, perch� solo le aziende di una certa dimensione avevano server interni e un sistema informatizzato raggiungibile dell'esterno. Oggi non esiste azienda che non abbia un calcolatore collegato ad internet, nella minima delle ipotesi. Nella stragrande maggioranza dei casi invece si hanno uffici con reti di computer, sedi dislocate per il mondo collegate tra di loro tramite lan, server interni per rendere disponibile materiale a consulenti e molto altro ancora. Si compirebbe un madornale errore pensando che se una azienda non tratta materiale sensibile allora non � in pericolo. Per esempio aziende concorrenti potrebbero voler rubare tecnologie e brevetti aziendali, consultare liste clienti e fornitori, male intenzionati potrebbero infettare le macchine per usarle come zombie nelle proprie botnet.

Oggi pi� che mai bisogna rendere consapevoli le aziende della situazione nella quale il mondo dell'informatica si sta evolvendo. Oltre che con l'informazione questo � possibile sviluppando software sicuro e proponendo soluzioni in tale settore. 

Darkcloud � un software di condivisione file, cio� un programma che installato su due computer e almeno un server permette ad ogni utente di salvare file sul server e condividere il file con l'altro utente. 

La condizione necessaria che questo software deve soddisfare � proteggere il file da utenti male intenzionati che potrebbero volersene appropriare. 

La soluzione a questo problema � garantita da tre fattori, frammentazione del file, crittografia e cloud computing. 

Prima di realizzare il software per� bisogna studiare le soluzioni che al momento sono adottate, verificarne i punti deboli e i pregi. Avendo questa conoscenza si pu� passare a progettare la struttura del software e la topologia della rete che esso sorregger�. Dopo aver delineato le linee guida potremo passare alla realizzazione del codice sorgente. Una volta ottenuto un programma funzionante � necessario testarlo per verificarne l'adeguatezza alle richieste.

Questi sono i passi che affronteremo nel corso della tesi, dall'analisi ai test. 
