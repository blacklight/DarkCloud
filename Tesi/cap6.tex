\chapter{Conclusioni}
Questa tesi ha voluto analizzare la nascita del progetto Darkcloud nelle fasi di studio, di analisi, di realizzazione della struttura di rete e implementazione delle funzioni avanzate. 

La prima fase quella di analisi ha portato alla conoscenza di molti aspetti che sono tornati utili per il progetto. Ad esempio avere una rete decentralizzata � utile per la sicurezza e la stabilit�, ma avere nodi che fungano sia da server che da client come nel p2p puro porta a rallentamenti delle prestazioni se si vuole un buon grado di anonimato ed espone a molte vulnerabilit�. Ancora avendo compreso il funzionamento delle DHT e il meccanismo di indicizzazione di Freenet � parso chiaro che per un software con un bacino di utenza piccolo la difficolt� di programmazione sarebbe stata troppo elevata. Un importante decisione � stata quella di optare per una rete statica, dove i nodi vanno aggiunti manualmente, � vero che sarebbe pi� comodo un modello dinamico, ma come insegnano le reti friend to friend questo va a discapito della sicurezza. Capire come la topologia di una rete influisce sulla scalabilit� delle prestazioni ha imposto l'adozione di due tipi di nodi con compiti ben distinti.

Per poter delineare la struttura ad alto livello del software si � dovuto valutare bene le caratteristiche che questo doveva avere. Ogni decisione in fase di progetto � un compromesso, bisogna valutare cosa � meglio in base alle esigenze dell'utente ultimo. La scelta del linguaggio Java ha il vantaggio della portabilit� e della disponibilit� di librerie molto fornite, per� essendo un linguaggio indipendente dalla piattaforma ha prestazioni minori di C++. Darkcloud dovendo funzionare su personal computer non ha problemi di limitatezza di risorse, questo ha fatto scegliere Java. Decidere che tipo di interfaccia usare � un importante scelta, chiaramente se si fosse optato per un eseguibile con interfaccia integrata si sarebbe ottenuta una maggior velocit� di sviluppo, ma un eseguibile che si comandi tramite terminali esterni sar� un grande vantaggio quando dovr� essere utilizzato da una utenza diversificata. Avere un terminale su pagina web, su tablet e altri terminali alternativi facilita ed estende l'uso di Darkcloud. In fase di progetto � anche stato scelto che tipo di comandi realizzare, l'uso di comandi semplici o articolati, favorendo comandi semplici si � data l'opportunit� a futuri sviluppatori di interfacce di realizzare comandi pi� complessi e articolati.

Una volta che le specifiche del codice erano ormai nitide si � proceduto ad un lavoro in team. La prima parte della programmazione ha riguardato la struttura di base con la definizione dei diversi tipi di entit� e dei meccanismi di comunicazione tra di essi, quindi la definizione di un protocollo. Una volta che la base � stata funzionante si � passato alla realizzazione delle funzioni tramite le quali i nodi eseguivano tra di loro procedimenti complessi. Non sono mancate le difficolt�, una delle maggiori riguarda la crittografia. Infatti si � presentato l'ostacolo del formato delle chiavi e dei dati criptati complicato dal fatto che tutti i metodi dialogano tra di loro con dati in base 64. Ma a parte queste piccolezze l'aver progettato fin dall'inizio un buon protocollo e aver per prima cosa scritto una struttura del programma prestando attenzione all'incapsulamento dei dati e alla modularit� ha semplificato notevolmente l'implementazione di funzioni complesse.

La fase di test dell'applicazione infine ha rivelato che il software reagisce bene ad un uso intenso. Anche se i test sono stati svolti su una sola macchina per volta, quindi la potenza disponibile era molto meno di quella che una rete metter� a disposizione, i risultati fanno capire che la scalabilit� del carico di lavoro � ottima. I tempi brevi garantiscono che in un caso di utilizzo reale con molti nodi funzionanti i tempi di attesa riguarderanno le latenze di rete e non tempi di software. Le percentuali di utilizzo della cpu rivelano anche che il carico di lavoro � davvero minimo. 

Alcuni sviluppi futuri tramite cui si potrebbe far crescere questo progetto sono lo studio di librerie di alcuni servizi di cloud computing per realizzare versioni ad hoc di Darkcloud da eseguire a livello SaaS, potrebbe essere utile aggiungere alla funzione share la possibilit� di accettare condivisioni anche quando il destinatario � offline tramite una lista di condivisioni in attesa, lo sviluppo di terminali che permettano di usare Darkcloud su tablet o smartphone infine sarebbe utile implementare un codice di correzione degli errori che si occupi di ricostruire dati danneggiati nelle comunicazioni. 

Il programma Darkcloud � un ottima base per reti di condivisione dati in modo sicuro, e senza dubbio sar� molto utile a chi svilupper� reti simili in futuro. Il codice � rilasciato in licenza GNU GPL 3 e quindi potr� essere riutilizzato tutto o in parte da altri utenti, non che migliorato o modificato dagli utilizzatori. 

Rappresenta una novit� nel settore usando la frammentazione delle informazioni su una rete anonima in abbinamento al cloud computing per il mantenimento delle stesse in sicurezza.

Spero che questo trattato vi abbia illuminato sul funzionamento del software e possa aiutarvi sia a comprenderne le meccaniche sia a guidarvi nell'utilizzo.
\\
\begin{center}\textsl{Gionatan Fortunato}\end{center}