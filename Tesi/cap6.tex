\chapter{Conclusioni}
\section{Sviluppi futuri}
Nonostante il programma sia pienamente funzionante, e i nodi server possano gi� essere installati su sistemi operativi virtuali situati su piattaforme di cloud computing, il progetto iniziale prevede che il software sia ottimizzato per il cloud computing. Come avevamo accennato al termine del capitolo 2 esistono servizi cloud di tre livelli: IaaS quindi hardware remoto, PaaS cio� piattaforme remote e SaaS cio� software remoto. Al momento Darkcloud pu� essere usato su servizi cloud di hardware e piattaforma remoti. Il prossimo passo sar� verso l'utilizzo di un livello ancora pi� alto, cio� SaaS. Se si utilizza un servizio per avere direttamente un software funzionante in remoto invece che dover realizzare un sistema operativo o una piattaforma sulla quale far girare il nostro software si sprecheranno meno risorse e questo comporter� un costo minore delle istanze server di Darkcloud, potendo aumentarne cosi il numero a favore della sicurezza. Avere software ottimizzato ed eseguito su sistemi pi� embedded che general purpose rende pi� sicuri i nostri dati e i nostri nodi server. Ogni provider di cloud ha librerie diverse per ottimizzate l'esecuzione del codice sui suoi sistemi. 

Quindi per poter utilizzare servizi cloud a livello SaaS bisogner� studiarne le librerie e andarle ad inserire nel codice per realizzare versioni di Darkcloud ad hoc per i nodi server. Dato che era prevista questa evoluzione il software � stato strutturato in modo da rendere semplice l'implementazione di nuove librerie. 

Dato che si potr� cosi sviluppare una architettura ancora pi� solida e complessa con l'utilizzo di vari servizi cloud si potrebbe anche introdurre un algoritmo di replicazione parziale dei dati in modo da avere una ridondanza dei dati su pi� servizi. Questo potrebbe garantire non solo una maggior garanzia sulla segretezza dei propri dati, ma anche una maggior reperibilit� dei dati. Non dovendo temere l'eventualit� di disservizi da parte di alcuni provider di cloud. 

Al momento quando il un dato viene trasmesso o ricevuto vengono fatti dei controlli per testare l'eventuale presenza di errori nel dato. Questi controlli si basano sul checksum e possono garantire solo l'integrit� del dato. Un interessante apporto al software potrebbe essere dato dall'introduzione di un algoritmo di correzione degli errori. Esistono infatti algoritmi che trasmettendo i dati in frammenti, in caso di perdita di un frammento, sono in grado di ricostruirlo in base agli altri.

Oltre ad adattare Darkcloud a vari servizi cloud uno sviluppo futuro potrebbe riguardare la realizzazione di terminali. Con terminali si intende software di interfaccia tra la rete Darkcloud e l'utente, che al momento � svolto dal nostro script python client.py. L'unico limite allo sviluppo dei terminali � la fantasia, si potrebbe realizzare interfacce su pagine web, software per personal computer, applicazioni per tablet smartphone e cellulari. Dato che questo software � stato realizzato puntando alla sicurezza dei dati, vogliamo ricordare ai futuri sviluppatori di terminali di tenere sempre presente questo, in modo da rinforzare la sicurezza nel punto pi� vulnerabile delle moderne tecnologie di rete che � vicino all'utente utilizzatore, che nella maggior parte dei casi � il meno esperto.

\section{Considerazioni finali}
Questa tesi ha voluto analizzare la nascita del progetto Darkcloud, il tesista ha assistito l'ingegner Fabio Manganiello e lo ha aiutato nelle fasi di studio, di analisi, di realizzazione della struttura di rete e implementazione delle funzioni avanzate. 

Nel corso di questa tesi si � potuto conoscere e approfondire la programmazione in java orientata alle reti; le tecniche gli strumenti e i modi di programmazione in team. Inoltre � stato molto istruttivo poter seguire un progetto dal confronto con la richiesta fino ai test sulle performance. Senza dubbio il tesista ha assimilato molto dal lavorare a stretto contatto con un ingegnere esperto di programmazione con un background e una conoscenza dell'informatica molto vasta quale � stato l'ingegner Fabio Manganiello. 

Il tesista ha potuto approfondire la conoscenza di molte strutture di rete in uso oggi e che segneranno il futuro del web, non che la nascente tecnologia del cloud computing che sar� sempre pi� utilizzata.

Il programma Darkcloud � un ottima base per reti di condivisione dati in modo sicuro, e senza dubbio sar� molto utile a chi svilupper� reti simili in futuro. Il codice � rilasciato in licenza GNU GPL 3 e quindi potr� essere riutilizzato tutto o in parte da altri utenti, non che migliorato o modificato dagli utilizzatori. 

Rappresenta una novit� nel settore in quanto non � mai stato usato un approccio che utilizzasse la frammentazione delle informazioni per il mantenimento delle informazioni in sicurezza. Inoltre anche l'architettura di rete e l'inserimento del cloud computing sono poco usati in questo tipo di software.

Si ringrazia il professor Colajanni Michele per l'opportunit� di partecipare ad un cos� interessante progetto, inoltre un ringraziamento va anche all'ingegner Fabio Manganiello per il sostegno e l'aiuto dato al tesista.

Spero che questo trattato vi abbiamo illuminato sul funzionamento del software e possa aiutarvi sia a comprenderne le meccaniche sia a guidarvi nell'utilizzo.
\\
\begin{center}\textsl{Gionatan Fortunato}\end{center}